\def\Abstract{
% hallo,
% \begin{itemize}
%     \item existujou set funkce
%     \item set funkce se používaj....
%     \item problém: hodně subsetů, ale! získat value subsetu může být hard
%     \item když prostě nejde mít všechno, tak se spokojíme s tím, že teda máme jen něco
%     \item řešení: najít to něco co je nejlepší mít
% \end{itemize}
% hence, in conclusin, vyřešili jsme život?! HA! tak já už musim. filip
	Set functions offer a way to express the relationship between subsets of some finite ground set.
	This is used in countless fields, including explainable AI, combinatorial auctions, and cooperative game theory.
	However, when applying set functions to a real-world problem, there is a significant roadblock: their size grows exponentially in size of the ground set, while finding out the value of even a single subset might be hard---costing, e.g., money, time, or computational power.
	In this thesis, we present a framework for striking a balance between the resources we need to expend, and the amount of information we learn about the set function.
	We frame this as an optimization problem, for which we find both exact solutions as well as approximations via reinforcement learning.
	We establish a measure for the ambiguity arising from the unknown values and study its properties.
	We show the performance of our approaches on simple instances of the problem as well as on the very general class of supermodular functions.
	Further, we define a very simple heuristic which drastically decreases our ambiguity metric on the supermodular class while only requiring a linear number of values to be known.
}
