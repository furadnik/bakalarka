%%% A template for a simple PDF/A file like a stand-alone abstract of the thesis.

\documentclass[12pt]{report}

\usepackage[a4paper, hmargin=1in, vmargin=1in]{geometry}
\usepackage[a-2u]{pdfx}
\usepackage[utf8]{inputenc}
\usepackage[T1]{fontenc}
\usepackage{lmodern}
\usepackage{textcomp}

\begin{document}

Set funkce nabízejí způsob, jak vyjádřit vztah mezi podmnožinami konečné množiny.
Používají se v mnoha oblastech, včetně vysvětlitelné umělé inteligence, kombinatorických aukcí a kooperativní teorie her.
Při aplikaci množinových funkcí na problém v reálném světě však existuje značná překážka: jejich velikost roste exponenciálně s velikostí nosné množiny.
Žjištění hodnoty byť jediné podmnožiny však může být obtížné –- stojí to například peníze, čas nebo výpočetní výkon.
V této práci představujeme způsob jak nalézt rovnováhu mezi zdroji, které musíme vynaložit, a množstvím informací, které se o set funkci zjistíme.
Pohlížíme na toto jako na optimalizační problém, pro který představujeme jak přesná řešení, tak aproximace pomocí zpětnovazebního učení.
Definujeme míru nejednoznačnosti, která vzniká díky neznámým hodnotám, a studujeme její vlastnosti.
Studujeme naše algoritmy na jednoduchých příkladech problému, stejně jako na velmi obecné třídě supermodulárních funkcí.
Dále definujeme jednoduchou heuristiku, která drasticky snižuje naši metriku nejednoznačnosti na supermodulárních funkcích, při znalosti pouze lineárního počtu hodnot.

\end{document}
